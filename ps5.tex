% Options for packages loaded elsewhere
\PassOptionsToPackage{unicode}{hyperref}
\PassOptionsToPackage{hyphens}{url}
\PassOptionsToPackage{dvipsnames,svgnames,x11names}{xcolor}
%
\documentclass[
  letterpaper,
  DIV=11,
  numbers=noendperiod]{scrartcl}

\usepackage{amsmath,amssymb}
\usepackage{iftex}
\ifPDFTeX
  \usepackage[T1]{fontenc}
  \usepackage[utf8]{inputenc}
  \usepackage{textcomp} % provide euro and other symbols
\else % if luatex or xetex
  \usepackage{unicode-math}
  \defaultfontfeatures{Scale=MatchLowercase}
  \defaultfontfeatures[\rmfamily]{Ligatures=TeX,Scale=1}
\fi
\usepackage{lmodern}
\ifPDFTeX\else  
    % xetex/luatex font selection
\fi
% Use upquote if available, for straight quotes in verbatim environments
\IfFileExists{upquote.sty}{\usepackage{upquote}}{}
\IfFileExists{microtype.sty}{% use microtype if available
  \usepackage[]{microtype}
  \UseMicrotypeSet[protrusion]{basicmath} % disable protrusion for tt fonts
}{}
\makeatletter
\@ifundefined{KOMAClassName}{% if non-KOMA class
  \IfFileExists{parskip.sty}{%
    \usepackage{parskip}
  }{% else
    \setlength{\parindent}{0pt}
    \setlength{\parskip}{6pt plus 2pt minus 1pt}}
}{% if KOMA class
  \KOMAoptions{parskip=half}}
\makeatother
\usepackage{xcolor}
\setlength{\emergencystretch}{3em} % prevent overfull lines
\setcounter{secnumdepth}{-\maxdimen} % remove section numbering
% Make \paragraph and \subparagraph free-standing
\makeatletter
\ifx\paragraph\undefined\else
  \let\oldparagraph\paragraph
  \renewcommand{\paragraph}{
    \@ifstar
      \xxxParagraphStar
      \xxxParagraphNoStar
  }
  \newcommand{\xxxParagraphStar}[1]{\oldparagraph*{#1}\mbox{}}
  \newcommand{\xxxParagraphNoStar}[1]{\oldparagraph{#1}\mbox{}}
\fi
\ifx\subparagraph\undefined\else
  \let\oldsubparagraph\subparagraph
  \renewcommand{\subparagraph}{
    \@ifstar
      \xxxSubParagraphStar
      \xxxSubParagraphNoStar
  }
  \newcommand{\xxxSubParagraphStar}[1]{\oldsubparagraph*{#1}\mbox{}}
  \newcommand{\xxxSubParagraphNoStar}[1]{\oldsubparagraph{#1}\mbox{}}
\fi
\makeatother

\usepackage{color}
\usepackage{fancyvrb}
\newcommand{\VerbBar}{|}
\newcommand{\VERB}{\Verb[commandchars=\\\{\}]}
\DefineVerbatimEnvironment{Highlighting}{Verbatim}{commandchars=\\\{\}}
% Add ',fontsize=\small' for more characters per line
\usepackage{framed}
\definecolor{shadecolor}{RGB}{241,243,245}
\newenvironment{Shaded}{\begin{snugshade}}{\end{snugshade}}
\newcommand{\AlertTok}[1]{\textcolor[rgb]{0.68,0.00,0.00}{#1}}
\newcommand{\AnnotationTok}[1]{\textcolor[rgb]{0.37,0.37,0.37}{#1}}
\newcommand{\AttributeTok}[1]{\textcolor[rgb]{0.40,0.45,0.13}{#1}}
\newcommand{\BaseNTok}[1]{\textcolor[rgb]{0.68,0.00,0.00}{#1}}
\newcommand{\BuiltInTok}[1]{\textcolor[rgb]{0.00,0.23,0.31}{#1}}
\newcommand{\CharTok}[1]{\textcolor[rgb]{0.13,0.47,0.30}{#1}}
\newcommand{\CommentTok}[1]{\textcolor[rgb]{0.37,0.37,0.37}{#1}}
\newcommand{\CommentVarTok}[1]{\textcolor[rgb]{0.37,0.37,0.37}{\textit{#1}}}
\newcommand{\ConstantTok}[1]{\textcolor[rgb]{0.56,0.35,0.01}{#1}}
\newcommand{\ControlFlowTok}[1]{\textcolor[rgb]{0.00,0.23,0.31}{\textbf{#1}}}
\newcommand{\DataTypeTok}[1]{\textcolor[rgb]{0.68,0.00,0.00}{#1}}
\newcommand{\DecValTok}[1]{\textcolor[rgb]{0.68,0.00,0.00}{#1}}
\newcommand{\DocumentationTok}[1]{\textcolor[rgb]{0.37,0.37,0.37}{\textit{#1}}}
\newcommand{\ErrorTok}[1]{\textcolor[rgb]{0.68,0.00,0.00}{#1}}
\newcommand{\ExtensionTok}[1]{\textcolor[rgb]{0.00,0.23,0.31}{#1}}
\newcommand{\FloatTok}[1]{\textcolor[rgb]{0.68,0.00,0.00}{#1}}
\newcommand{\FunctionTok}[1]{\textcolor[rgb]{0.28,0.35,0.67}{#1}}
\newcommand{\ImportTok}[1]{\textcolor[rgb]{0.00,0.46,0.62}{#1}}
\newcommand{\InformationTok}[1]{\textcolor[rgb]{0.37,0.37,0.37}{#1}}
\newcommand{\KeywordTok}[1]{\textcolor[rgb]{0.00,0.23,0.31}{\textbf{#1}}}
\newcommand{\NormalTok}[1]{\textcolor[rgb]{0.00,0.23,0.31}{#1}}
\newcommand{\OperatorTok}[1]{\textcolor[rgb]{0.37,0.37,0.37}{#1}}
\newcommand{\OtherTok}[1]{\textcolor[rgb]{0.00,0.23,0.31}{#1}}
\newcommand{\PreprocessorTok}[1]{\textcolor[rgb]{0.68,0.00,0.00}{#1}}
\newcommand{\RegionMarkerTok}[1]{\textcolor[rgb]{0.00,0.23,0.31}{#1}}
\newcommand{\SpecialCharTok}[1]{\textcolor[rgb]{0.37,0.37,0.37}{#1}}
\newcommand{\SpecialStringTok}[1]{\textcolor[rgb]{0.13,0.47,0.30}{#1}}
\newcommand{\StringTok}[1]{\textcolor[rgb]{0.13,0.47,0.30}{#1}}
\newcommand{\VariableTok}[1]{\textcolor[rgb]{0.07,0.07,0.07}{#1}}
\newcommand{\VerbatimStringTok}[1]{\textcolor[rgb]{0.13,0.47,0.30}{#1}}
\newcommand{\WarningTok}[1]{\textcolor[rgb]{0.37,0.37,0.37}{\textit{#1}}}

\providecommand{\tightlist}{%
  \setlength{\itemsep}{0pt}\setlength{\parskip}{0pt}}\usepackage{longtable,booktabs,array}
\usepackage{calc} % for calculating minipage widths
% Correct order of tables after \paragraph or \subparagraph
\usepackage{etoolbox}
\makeatletter
\patchcmd\longtable{\par}{\if@noskipsec\mbox{}\fi\par}{}{}
\makeatother
% Allow footnotes in longtable head/foot
\IfFileExists{footnotehyper.sty}{\usepackage{footnotehyper}}{\usepackage{footnote}}
\makesavenoteenv{longtable}
\usepackage{graphicx}
\makeatletter
\def\maxwidth{\ifdim\Gin@nat@width>\linewidth\linewidth\else\Gin@nat@width\fi}
\def\maxheight{\ifdim\Gin@nat@height>\textheight\textheight\else\Gin@nat@height\fi}
\makeatother
% Scale images if necessary, so that they will not overflow the page
% margins by default, and it is still possible to overwrite the defaults
% using explicit options in \includegraphics[width, height, ...]{}
\setkeys{Gin}{width=\maxwidth,height=\maxheight,keepaspectratio}
% Set default figure placement to htbp
\makeatletter
\def\fps@figure{htbp}
\makeatother

\usepackage{fvextra}
\DefineVerbatimEnvironment{Highlighting}{Verbatim}{breaklines,commandchars=\\\{\}}
\KOMAoption{captions}{tableheading}
\makeatletter
\@ifpackageloaded{caption}{}{\usepackage{caption}}
\AtBeginDocument{%
\ifdefined\contentsname
  \renewcommand*\contentsname{Table of contents}
\else
  \newcommand\contentsname{Table of contents}
\fi
\ifdefined\listfigurename
  \renewcommand*\listfigurename{List of Figures}
\else
  \newcommand\listfigurename{List of Figures}
\fi
\ifdefined\listtablename
  \renewcommand*\listtablename{List of Tables}
\else
  \newcommand\listtablename{List of Tables}
\fi
\ifdefined\figurename
  \renewcommand*\figurename{Figure}
\else
  \newcommand\figurename{Figure}
\fi
\ifdefined\tablename
  \renewcommand*\tablename{Table}
\else
  \newcommand\tablename{Table}
\fi
}
\@ifpackageloaded{float}{}{\usepackage{float}}
\floatstyle{ruled}
\@ifundefined{c@chapter}{\newfloat{codelisting}{h}{lop}}{\newfloat{codelisting}{h}{lop}[chapter]}
\floatname{codelisting}{Listing}
\newcommand*\listoflistings{\listof{codelisting}{List of Listings}}
\makeatother
\makeatletter
\makeatother
\makeatletter
\@ifpackageloaded{caption}{}{\usepackage{caption}}
\@ifpackageloaded{subcaption}{}{\usepackage{subcaption}}
\makeatother

\ifLuaTeX
  \usepackage{selnolig}  % disable illegal ligatures
\fi
\usepackage{bookmark}

\IfFileExists{xurl.sty}{\usepackage{xurl}}{} % add URL line breaks if available
\urlstyle{same} % disable monospaced font for URLs
\hypersetup{
  pdftitle={PS5 Andy Fan Will Sigal},
  colorlinks=true,
  linkcolor={blue},
  filecolor={Maroon},
  citecolor={Blue},
  urlcolor={Blue},
  pdfcreator={LaTeX via pandoc}}


\title{PS5 Andy Fan Will Sigal}
\author{}
\date{}

\begin{document}
\maketitle

\RecustomVerbatimEnvironment{verbatim}{Verbatim}{
  showspaces = false,
  showtabs = false,
  breaksymbolleft={},
  breaklines
}


\textbf{PS4:} Due Sat Nov 9 at 5:00PM Central. Worth 100 points.

\subsection{Style Points (10 pts)}\label{style-points-10-pts}

\subsection{Submission Steps (10 pts)}\label{submission-steps-10-pts}

\begin{enumerate}
\def\labelenumi{\arabic{enumi}.}
\tightlist
\item
  This problem set is a paired problem set.
\item
  Play paper, scissors, rock to determine who goes first. Call that
  person \emph{Partner 1}.

  \begin{itemize}
  \tightlist
  \item
    Partner 1 (name and cnet ID): Andy Fan, fanx
  \item
    Partner 2 (name and cnet ID): Will Sigal, Wsigal
  \end{itemize}
\item
  Partner 1 will accept the \texttt{ps5} and then share the link it
  creates with their partner. You can only share it with one partner so
  you will not be able to change it after your partner has accepted.
\item
  ``This submission is our work alone and complies with the 30538
  integrity policy.'' Add your initials to indicate your agreement:
\end{enumerate}

AF WS

\begin{enumerate}
\def\labelenumi{\arabic{enumi}.}
\setcounter{enumi}{4}
\tightlist
\item
  ``I have uploaded the names of anyone else other than my partner and I
  worked with on the problem set
  \textbf{\href{https://docs.google.com/forms/d/185usrCREQaUbvAXpWhChkjghdGgmAZXA3lPWpXLLsts/edit}{here}}''
  (1 point)
\item
  Late coins used this pset: (Andy:0); (Will:0) Late coins left after
  submission: (Andy: 3) ; (Will: 4)
\item
  Knit your \texttt{ps5.qmd} to an PDF file to make \texttt{ps5.pdf},

  \begin{itemize}
  \tightlist
  \item
    The PDF should not be more than 25 pages. Use \texttt{head()} and
    re-size figures when appropriate.
  \end{itemize}
\item
  (Partner 1): push \texttt{ps5.qmd} and \texttt{ps5.pdf} to your github
  repo.
\item
  (Partner 1): submit \texttt{ps5.pdf} via Gradescope. Add your partner
  on Gradescope.
\item
  (Partner 1): tag your submission in Gradescope
\end{enumerate}

\begin{Shaded}
\begin{Highlighting}[]
\CommentTok{\#\#\# SETUP }
\ImportTok{import}\NormalTok{ pandas }\ImportTok{as}\NormalTok{ pd}
\ImportTok{import}\NormalTok{ altair }\ImportTok{as}\NormalTok{ alt}
\ImportTok{import}\NormalTok{ time}
\ImportTok{import}\NormalTok{ os}
\ImportTok{import}\NormalTok{ warnings}
\ImportTok{import}\NormalTok{ geopandas }\ImportTok{as}\NormalTok{ gpd}
\ImportTok{import}\NormalTok{ numpy }\ImportTok{as}\NormalTok{ np}
\ImportTok{import}\NormalTok{ matplotlib.pyplot }\ImportTok{as}\NormalTok{ plt}
\NormalTok{warnings.filterwarnings(}\StringTok{\textquotesingle{}ignore\textquotesingle{}}\NormalTok{)}
\ImportTok{import}\NormalTok{ requests}
\ImportTok{from}\NormalTok{ bs4 }\ImportTok{import}\NormalTok{ BeautifulSoup}
\end{Highlighting}
\end{Shaded}

\subsection{(30 points) Step 1: Develop initial scraper and
crawler}\label{points-step-1-develop-initial-scraper-and-crawler}

\paragraph{1. (Partner 1) Scraping: Go to the first page of the HHS
OIG's ``Enforcement Actions''page and scrape and collect the following
into a dataset: • Title of the enforcement action • Date • Category
(e.g, ``Criminal and Civil Actions'') • Link associated with the
enforcement action Collect your output into a tidy dataframe and print
its
head.}\label{partner-1-scraping-go-to-the-first-page-of-the-hhs-oigs-enforcement-actionspage-and-scrape-and-collect-the-following-into-a-dataset-title-of-the-enforcement-action-date-category-e.g-criminal-and-civil-actions-link-associated-with-the-enforcement-action-collect-your-output-into-a-tidy-dataframe-and-print-its-head.}

\begin{Shaded}
\begin{Highlighting}[]
\CommentTok{\#\#\# making soup}
\NormalTok{url1 }\OperatorTok{=} \StringTok{\textquotesingle{}https://oig.hhs.gov/fraud/enforcement\textquotesingle{}}
\NormalTok{response1 }\OperatorTok{=}\NormalTok{ requests.get(url1)}
\NormalTok{soup1 }\OperatorTok{=}\NormalTok{ BeautifulSoup(response1.content, }\StringTok{\textquotesingle{}lxml\textquotesingle{}}\NormalTok{)}
\end{Highlighting}
\end{Shaded}

\begin{Shaded}
\begin{Highlighting}[]
\CommentTok{\#\#\# find title of enforcements}
\NormalTok{li\_blocks }\OperatorTok{=}\NormalTok{ soup1.find\_all(}\StringTok{\textquotesingle{}h2\textquotesingle{}}\NormalTok{) }\CommentTok{\#h2 classes with nested \textquotesingle{}a\textquotesingle{} titles. li\_blocks[2:21] are the 20 ones}
\NormalTok{li\_titles }\OperatorTok{=}\NormalTok{ []}
\ControlFlowTok{for}\NormalTok{ h2 }\KeywordTok{in}\NormalTok{ li\_blocks:}
    \CommentTok{\# Find all \textquotesingle{}a\textquotesingle{} tags within each \textquotesingle{}h2\textquotesingle{} element}
    \ControlFlowTok{for}\NormalTok{ a\_tag }\KeywordTok{in}\NormalTok{ h2.find\_all(}\StringTok{\textquotesingle{}a\textquotesingle{}}\NormalTok{):}
\NormalTok{        li\_titles.append(a\_tag)}
\NormalTok{li\_titles[}\DecValTok{0}\NormalTok{:}\DecValTok{5}\NormalTok{]}
\NormalTok{df\_title }\OperatorTok{=}\NormalTok{ pd.DataFrame(li\_titles) }\CommentTok{\# dataframe with titles}
\NormalTok{df\_title.columns }\OperatorTok{=}\NormalTok{ [}\StringTok{\textquotesingle{}title\textquotesingle{}}\NormalTok{]}
\end{Highlighting}
\end{Shaded}

Title: each title is a h2/href(URL) class, under h2 class. `h2 class'
under `header class' under `div class' under `li class'

\begin{Shaded}
\begin{Highlighting}[]
\CommentTok{\#\#\# find date}
\NormalTok{span\_blocks }\OperatorTok{=}\NormalTok{ soup1.find\_all(}\StringTok{\textquotesingle{}span\textquotesingle{}}\NormalTok{, attrs}\OperatorTok{=}\NormalTok{\{}\StringTok{\textquotesingle{}class\textquotesingle{}}\NormalTok{: }\StringTok{\textquotesingle{}text{-}base{-}dark padding{-}right{-}105\textquotesingle{}}\NormalTok{\})}
\NormalTok{span\_blocks[}\DecValTok{0}\NormalTok{:}\DecValTok{5}\NormalTok{]}
\NormalTok{df\_date }\OperatorTok{=}\NormalTok{ pd.DataFrame(span\_blocks) }
\NormalTok{df\_date.columns }\OperatorTok{=}\NormalTok{ [}\StringTok{\textquotesingle{}date\textquotesingle{}}\NormalTok{]}
\CommentTok{\#asked chat gpt \textquotesingle{}how to i search for \textquotesingle{}span\textquotesingle{} class with attribute xxx\textquotesingle{}}
\end{Highlighting}
\end{Shaded}

Date: is under span class

\begin{Shaded}
\begin{Highlighting}[]
\CommentTok{\#\#\# find category}
\NormalTok{li\_blocks\_dt }\OperatorTok{=}\NormalTok{ soup1.find\_all(}\StringTok{\textquotesingle{}li\textquotesingle{}}\NormalTok{, attrs}\OperatorTok{=}\NormalTok{\{}\StringTok{\textquotesingle{}class\textquotesingle{}}\NormalTok{: }\StringTok{\textquotesingle{}display{-}inline{-}block usa{-}tag text{-}no{-}lowercase text{-}base{-}darkest bg{-}base{-}lightest margin{-}right{-}1\textquotesingle{}}\NormalTok{\})}
\NormalTok{li\_blocks[}\DecValTok{0}\NormalTok{:}\DecValTok{5}\NormalTok{]}
\NormalTok{df\_category }\OperatorTok{=}\NormalTok{ pd.DataFrame(li\_blocks\_dt)}
\NormalTok{df\_category.columns }\OperatorTok{=}\NormalTok{ [}\StringTok{\textquotesingle{}category\textquotesingle{}}\NormalTok{]}
\end{Highlighting}
\end{Shaded}

Category: each title is a li class.'l1 class' under `ul class' under
`div class' under `header class'

\begin{Shaded}
\begin{Highlighting}[]
\CommentTok{\#\#\# find link}
\CommentTok{\#use the list of titles from p1 and extract href}
\NormalTok{link\_blocks }\OperatorTok{=}\NormalTok{ [link.get(}\StringTok{\textquotesingle{}href\textquotesingle{}}\NormalTok{) }\ControlFlowTok{for}\NormalTok{ link }\KeywordTok{in}\NormalTok{ li\_titles]}
\NormalTok{df\_link }\OperatorTok{=}\NormalTok{ pd.DataFrame(link\_blocks)}
\NormalTok{df\_link.columns }\OperatorTok{=}\NormalTok{ [}\StringTok{\textquotesingle{}link\textquotesingle{}}\NormalTok{]}
\NormalTok{df\_link[}\StringTok{\textquotesingle{}link\textquotesingle{}}\NormalTok{] }\OperatorTok{=} \StringTok{"https://oig.hhs.gov"} \OperatorTok{+}\NormalTok{ df\_link[}\StringTok{\textquotesingle{}link\textquotesingle{}}\NormalTok{]}
\end{Highlighting}
\end{Shaded}

Link: link=href class, under h2 class. `h2 class' under `header class'
under `div class' under `li class'

\begin{Shaded}
\begin{Highlighting}[]
\CommentTok{\#\#\# combine dataframes}
\NormalTok{df\_Q1 }\OperatorTok{=}\NormalTok{ pd.concat([df\_title, df\_date, df\_category, df\_link], axis}\OperatorTok{=}\DecValTok{1}\NormalTok{)}
\end{Highlighting}
\end{Shaded}

\paragraph{2. (Partner 1) Crawling: Then for each enforcement action,
click the link and collect the name of the agency involved (e.g., for
this link, it would be U.S. Attorney's Office, Eastern District of
Washington).}\label{partner-1-crawling-then-for-each-enforcement-action-click-the-link-and-collect-the-name-of-the-agency-involved-e.g.-for-this-link-it-would-be-u.s.-attorneys-office-eastern-district-of-washington.}

\subsection{(30 points) Step 2: Making the scraper
dynamic}\label{points-step-2-making-the-scraper-dynamic}

\paragraph{1. Turning the scraper into a function: You will write a
function that takes as input a month and a year, and then pulls and
formats the enforcement actions like in Step 1 starting from that
month+year to
today.}\label{turning-the-scraper-into-a-function-you-will-write-a-function-that-takes-as-input-a-month-and-a-year-and-then-pulls-and-formats-the-enforcement-actions-like-in-step-1-starting-from-that-monthyear-to-today.}

\paragraph{a}\label{a}

\paragraph{b}\label{b}

\paragraph{c}\label{c}

\subsection{(15 points) Step 3: Plot data based on scraped data (using
altair)}\label{points-step-3-plot-data-based-on-scraped-data-using-altair}

\paragraph{1. (Partner 2) Plot a line chart that shows: the number of
enforcement actions over time (aggregated to each month+year) overall
since January
2021,}\label{partner-2-plot-a-line-chart-that-shows-the-number-of-enforcement-actions-over-time-aggregated-to-each-monthyear-overall-since-january-2021}

\paragraph{2. (Partner 1) Plot a line chart that shows: the number of
enforcement actions split out by: • ``Criminal and Civil Actions''
vs.~``State Enforcement Agencies'' • Five topics in the ``Criminal and
Civil Actions'' category: ``Health Care Fraud'', ``Financial Fraud'',
``Drug Enforcement'', ``Bribery/Corruption'', and
``Other''.}\label{partner-1-plot-a-line-chart-that-shows-the-number-of-enforcement-actions-split-out-by-criminal-and-civil-actions-vs.-state-enforcement-agencies-five-topics-in-the-criminal-and-civil-actions-category-health-care-fraud-financial-fraud-drug-enforcement-briberycorruption-and-other.}

\subsection{(15 points) Step 4: Create maps of enforcement activity For
these questions, use this US Attorney District shapefile (link) and a
Census state shapefile
(link)}\label{points-step-4-create-maps-of-enforcement-activity-for-these-questions-use-this-us-attorney-district-shapefile-link-and-a-census-state-shapefile-link}

\paragraph{1. (Partner 1) Map by state: Among actions taken by
state-level agencies, clean the state names you collected and plot a
choropleth of the number of enforcement actions for each state. Hint:
look for ``State of'' in the agency
info!}\label{partner-1-map-by-state-among-actions-taken-by-state-level-agencies-clean-the-state-names-you-collected-and-plot-a-choropleth-of-the-number-of-enforcement-actions-for-each-state.-hint-look-for-state-of-in-the-agency-info}

\paragraph{2. (Partner 2) Map by district: Among actions taken by US
Attorney District-level agencies, clean the district names so that you
can merge them with the shapefile, and then plot a choropleth of the
number of enforcement actions in each US Attorney District. Hint: look
for ``District'' in the agency
info.}\label{partner-2-map-by-district-among-actions-taken-by-us-attorney-district-level-agencies-clean-the-district-names-so-that-you-can-merge-them-with-the-shapefile-and-then-plot-a-choropleth-of-the-number-of-enforcement-actions-in-each-us-attorney-district.-hint-look-for-district-in-the-agency-info.}

\subsection{(10 points) Extra credit: Calculate the enforcement actions
on a per-capita basis (Both partners can work
together)}\label{points-extra-credit-calculate-the-enforcement-actions-on-a-per-capita-basis-both-partners-can-work-together}

\paragraph{1. Use the zip code shapefile from the previous problem set
and merge it with zip code level population data. (Go to Census Data
Portal, select ``ZIP Code Tabulation Area'', check ``All 5-digit ZIP
Code Tabulation Areas within United States'', and under ``P1 TOTAL
POPULATION'' select ``2020: DEC Demographic and Housing
Characteristics''. Download the
csv.).}\label{use-the-zip-code-shapefile-from-the-previous-problem-set-and-merge-it-with-zip-code-level-population-data.-go-to-census-data-portal-select-zip-code-tabulation-area-check-all-5-digit-zip-code-tabulation-areas-within-united-states-and-under-p1-total-population-select-2020-dec-demographic-and-housing-characteristics.-download-the-csv..}

\paragraph{2. Conduct a spatial join between zip code shapefile and the
district shapefile, then aggregate to get population in each
district.}\label{conduct-a-spatial-join-between-zip-code-shapefile-and-the-district-shapefile-then-aggregate-to-get-population-in-each-district.}

\paragraph{3. Mapthe ratio of enforcement actions in each US Attorney
District. You can calculate the ratio by aggregating the number of
enforcement actions since January 2021 per district, and dividing it
with the population
data.}\label{mapthe-ratio-of-enforcement-actions-in-each-us-attorney-district.-you-can-calculate-the-ratio-by-aggregating-the-number-of-enforcement-actions-since-january-2021-per-district-and-dividing-it-with-the-population-data.}




\end{document}
